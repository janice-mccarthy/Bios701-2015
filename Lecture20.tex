
\documentclass[12pt]{article} % Use A4 paper with a 12pt font size - different paper sizes will require manual recalculation of page margins and border positions

% Generated with LaTeXDraw 2.0.8
% Mon Jun 17 19:00:40 EDT 2013
\usepackage[usenames,dvipsnames]{pstricks}
\usepackage{epsfig}
\usepackage{pst-grad} % For gradients
\usepackage{pst-plot} % For axes
\usepackage[left=1.3cm,right=4.6cm,top=1.8cm,bottom=4.0cm,marginparwidth=3.4cm]{geometry} % Adjust page margins
\usepackage{amsmath} % Required for equation customization
\usepackage{amssymb} % Required to include mathematical symbols
\usepackage{xcolor} % Required to specify colors by name
\usepackage{amsthm}
\usepackage{float}
\usepackage{tikz}
\usetikzlibrary{shapes,backgrounds,trees}
\usepackage{wasysym}

\makeatletter
\newcommand{\mytag}[2]{%
  \text{#1}%
  \@bsphack
  \protected@write\@auxout{}%
         {\string\newlabel{#2}{{#1}{\thepage}}}%
  \@esphack
}
\makeatother

\setlength{\parindent}{0cm} % Remove paragraph indentation
\newcommand{\tab}{\hspace*{2em}} % Defines a new command for some horizontal space
%\newcommand{\choose}[2]{\left(\begin{matrix}
%{#1}\\{#2}
%\end{matrix}\right)}
\date{}
\title{Introduction to Probability Theory - Lecture 20}
%----------------------------------------------------------------------------------------

\newtheorem{defn}{Definition}
\newtheorem{example}{Example}
\newtheorem{prop}{Proposition}
\newtheorem{exer}{Exercises}
\newtheorem{thm}{Therorem}
\begin{document}
\maketitle

\section{Joint Distributions - Continued}
Recall from last time, we defined the CDF, joint, marginal and conditional pmfs for jointly distributed discrete random varibles:
\begin{itemize}
\item joint pmf: $$f_{XY}(x,y) = P(X=x,Y=y)$$
\item joint CDF: $$F_{XY}(x,y) = P(X\leq x,Y\leq y) = \sum_{s\leq x}\sum_{t\leq y} f_{XY}(s,t)$$
\item marginal pmf of $X$:
$$f_X(x) = \sum_y f_{XY}(x,y)$$
and similarly, the marginal pmf of $Y$ is
$$f_{Y}(y) = \sum_x f_{XY}(x,y)$$  
\end{itemize}
And we wrote downa version of Bayes Rule and LOTP in terms of conditional and marginal pmfs. Bayes rule is:
$$f_{Y|X}(y) = \frac{f_{X|Y}(x) f_Y(y)}{f_X(x)}$$
and the LOTP is:
$$f_X(x) = \sum_y f_{X|Y})x) f_Y(y)$$
\subsection{Independence}
\begin{defn}
Two random variables $X$ and $Y$ are independent $\iff$ 
$$F_{XY}(x,y) = F_x(x)F_Y(y)$$
where $F_X(x)$ and $F_Y(y)$ are the marginal CDFs for $X$ and $Y$. An equivalent statement is that the joint pmf/pdf factors into the product of the marginals:
$$f_{XY}(x,y) = f_X(x)f_Y(y)$$
\end{defn} 
Note: we will see in some examples that it is not enough for $f_{XY}(x,y)$ to factor into two functions, one a function of $x$ and the other a function of $y$. This is because the variables may have a dependence based on the \emph{support} of the pmf/pdf.
\subsection{Examples of Discrete Joint Distributions}
We will consider two examples of multivariate discrete distributions: The exended hypergeometric distribution and the multinomial. As you may guess from the names, these are generalizations of the hypergeometric and binomial distributions.
\subsubsection{Extended }
\end{document}