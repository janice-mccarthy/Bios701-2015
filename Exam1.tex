
\documentclass{article}
\usepackage{amsmath}

\oddsidemargin=0in
\evensidemargin=0in
\textwidth=6.3in
\topmargin=-.5in
\textheight=9in

\parindent=0in
\pagestyle{empty}



%------------------------------------------------------------------
% PROBLEM, PART, AND POINT COUNTING...

% Create the problem number counter.  Initialize to zero.
\newcounter{problemnum}

% Specify that problems should be labeled with arabic numerals.
\renewcommand{\theproblemnum}{\arabic{problemnum}}


% Create the part-within-a-problem counter, "within" the problem counter.
% This counter resets to zero automatically every time the PROBLEMNUM counter
% is incremented.
\newcounter{partnum}[problemnum]

% Specify that parts should be labeled with lowercase letters.
\renewcommand{\thepartnum}{\alph{partnum}}

% Make a counter to keep track of total points assigned to problems...
\newcounter{totalpoints}

% Make counters to keep track of points for parts...
\newcounter{curprobpts}		% Points assigned for the problem as a whole.
\newcounter{totalparts}		% Total points assigned to the various parts.

% Make a counter to keep track of the number of points on each page...
\newcounter{pagepoints}
% This counter is reset each time a page is printed.

% This "program" keeps track of how many points appear on each page, so that
% the total can be printed on the page itself.  Points are added to the total
% for a page when the PART (not the problem) they are assigned to is specified.
% When a problem without parts appears, the PAGEPOINTS are incremented directly
% from the problem as a whole (CURPROBPTS).


%---------------------------------------------------------------------------


% The \problem environment first checks the information about the previous
% problem.  If no parts appeared (or if they were all assigned zero points,
% then it increments TOTALPOINTS directly from CURPROBPTS, the points assigned
% to the last problem as a whole.  If the last problem did contain parts, it
% checks to make sure that their point values total up to the correct sum.
% It then puts the problem number on the page, along with the points assigned
% to it.

\newenvironment{problem}[1]{
% STATEMENTS TO BE EXECUTED WHEN A NEW PROBLEM IS BEGUN:
%
% Increment the problem number counter, and set the current \ref value to that
% number.
\refstepcounter{problemnum}
%
% Add some vspace to separate from the last problem.
\vspace{0.15in} \par
%
\setcounter{curprobpts}{#1} \setcounter{totalparts}{0}	% Reset counters.
%
% Now put in the "announcement" on the page.
{\Large \bf \theproblemnum. \normalsize ({\it \arabic{curprobpts} point\null\ifnum \value{curprobpts} = 1\else s\fi}\/)}
}{
% STATEMENTS TO BE EXECUTED WHEN AN OLD PROBLEM IS ENDED:
%
% If no parts to problem, then increment TOTALPOINTS and PAGEPOINTS for the
% entire problem at once.
\ifnum \value{totalparts} = 0
	\addtocounter{totalpoints}{\value{curprobpts}}	% Add pts to total.
	\addtocounter{pagepoints}{\value{curprobpts}}	% Add pts to page total.
%
% If there were parts for the problem, then check to make sure they total up
% to the same number of points that the problem is worth. Issue a warning
% if not.
\else \ifnum \value{totalparts} = \value{curprobpts}
	\else \typeout{}
	\typeout{!!!!!!!   POINT ACCOUNTING ERROR   !!!!!!!!}
	\typeout{PROBLEM [\theproblemnum] WAS ALLOCATED \arabic{curprobpts} POINTS,}
	\typeout{BUT CONTAINS PARTS TOTALLING \arabic{totalparts} POINTS!}
	\typeout{}
	\fi
\fi
}


%---------------------------------------------------------------------------


% The \newpart command increments the part counter and displays an appropriate
% lowercase letter to mark the part.  It adds points to the point counter
% immediately.  If 0 points are specified, no point announcement is made.
% Otherwise, the announcement is in scriptsize italics.

\newcommand{\newpart}[1]
{
\refstepcounter{partnum}	% Set the current \ref value to the part number.
\hspace{0.25in}		% Indent the part by a quarter inch.
%
% If points are to be printed for this problem (signaled by point value > 0),
% then put them in in scriptsize italics.
\ifnum #1 > 0
	\makebox[0.5in][l]{{\bf \thepartnum.} {\bf ({\it #1 pt\ifnum #1 = 1\else s\fi\/}) \,\,}}
\else
	\makebox[0.25in][l]{({\bf \thepartnum})}
\fi
%
\hspace{0.1in}		% Lead the material away from the part "number".
%
\addtocounter{totalparts}{#1}	% Add points to totalparts for this problem.
\addtocounter{pagepoints}{#1}	% Add points to total for this page.
\addtocounter{totalpoints}{#1}	% Add points to total for entire test.
}


%---------------------------------------------------------------------------



% Just in case you want to skip some numbers in your test...

\newcommand{\skipproblem}[1]{\addtocounter{problemnum}{#1}}



%---------------------------------------------------------------------------


% The \showpoints command simply gives a count of the total points read in up to
% the location at which the command is placed.  Typically, one places one
% \showpoints command at the end of the latex file, just prior to the
% \end{document} command.  It can appear elsewhere, however.

\newcommand{\showpoints}
{
\typeout{}  
\typeout{====> A TOTAL OF \arabic{totalpoints} POINTS WERE READ.}
\typeout{}
}


%---------------------------------------------------------------------------



\begin{document}


% ooooooooooooooooooooooooooooooooooooooooooooooooooooooooooooooo
%                             COVER
% ooooooooooooooooooooooooooooooooooooooooooooooooooooooooooooooo 

\centerline{\huge \bf Midterm 1}        %%%(number of test)
\vfill \vfill
   
Biostat 701                               %%%(class number and section) 

10/21/2015 \hfill                             %%%(date of test)
{\bf Name: } $\underbrace{\hspace{2.7in}}_{\mbox{\tiny by writing my 
                                           name i swear by the honor code}}$
\vfill \vfill \vfill

{\bf Read all of the following information before starting the exam:}
\vspace{1pc}

\begin{itemize}                        %%%(change info. as desired)
	\item  Show all work, clearly and in order, if you want to get full
	credit.  I reserve the right to take off points if I cannot see how you 
	arrived at your answer (even if your final answer is correct).
	
	\item Justify your answers algebraically whenever possible to ensure 
	full credit. 
		
	\item  This test has 5 problems  %%%(number of problems)
	and is worth 100 points,           %%%(insert total number of points)  
	plus some extra credit at the end. 
	
	\item  Good luck!
\end{itemize}

\vfill \vfill \vfill

\clearpage

% problem with three parts

\begin{problem}{3}
Provide the name and pmf/pdf of the distribution of the following variables:
\begin{enumerate}
\item $X = $ number of independent $Bern(p)$ trials to obtain the first success.
\item $X = $ number of successes in $n$ independent $Bern(p)$ trials.
\item $X = $ number of independent $Bern(p)$ trials to obtain $r$ successes.
\end{enumerate}
\end{problem}
\begin{problem}{5}
Consider the following discrete function:
$$f(n) = \left\{
\begin{matrix}
 \frac{c}{n^3}& \textrm{ for } n=0,1,2,3,...\\
  0 &\textrm{ otherwise }
\end{matrix}\right.$$

Is there a value of $c$ that makes $f$ a valid pmf?
\end{problem}


\begin{problem}{20}
Consider the following (piece-wise) continuous function:
$$f(x) = \left\{
\begin{matrix}
 \frac{c}{x^4}& \textrm{ for } x>1\\
  0 &\textrm{ otherwise }
\end{matrix}\right.$$
\vspace{1pc}

\newpart{10}
Find the value of $c$ that makes this a valid pdf.
\vspace{1pc}

\newpart{10}
Find the CDF of a random variable $X$ with pdf given by $f$.
\end{problem}

\begin{problem}{20}
Let $X \sim \mathcal{N}(\mu,\sigma^2)$.  
\vspace{1pc}

\newpart{10}
Given that the $98^{th}$ percentile of the standard normal distribution is $z_{0.98} = 2.05$, compute the $98^{th}$ percentile of $X$, in terms of $\mu$ and $\sigma$.
\vspace{1pc}

\newpart{5}
Give an expression that represents the probability that $a<X<b$, for $a,b$ real numbers.
\vspace{1pc}

\newpart{5}
What is the probability that $X=\mu$?
\end{problem}
\vspace{1pc}

\begin{problem}{12}
Let $X$ and $Y$ be independent random variables with mean and variance $\mu_X,\sigma_X^2$ and $\mu_Y$ and $\sigma_Y^2$ respectively. Let
$$W= 4X-2Y +3$$
\vspace{1pc}

\newpart{5}
Compute $E(W)$.
\vspace{1pc}

\newpart{7}
Compute $Var(W)$.

\end{problem}

\begin{problem}{20}
Let $X \sim Unif(\left\{0,1,2,...,n\right\})$. Compute the mean and variance of $X$. \\\\

Note that:
$$\sum_{i=1}^n i = \frac{n(n+1)}2$$
$$\sum_{i=1}^n i^2 = \frac{n(n+1)(2n+1)}6$$

\end{problem}


\begin{problem}{20}
My gmail filter divides my email messages into 'Inbox', 'Social' and 'Promotions' (thank you, Google!). I estimate that 50\% of my messages are filtered into my Inbox, 20\% go into Social (no thank you, Facebook) and 30\% go into Promotions (go away, Macy's). I notice that about 90\% of the emails in Promotions contain the word 'free', whereas about 1\% of the messages in Inbox and 2\% of the messages in Social contain the word 'free'. A message comes in with the word 'free'. What is the probability that the message is filtered as Promotion?    
\vspace{1pc}
\end{problem}

% ooooooooooooooooooooooooooooooooooooooooooooooooooooooooooooooo
%                         BONUS & SURVEY
% ooooooooooooooooooooooooooooooooooooooooooooooooooooooooooooooo

{\large \bf Bonus Question (2 Extra Credit Points):}
\vspace{1pc}

BONUS QUESTION  
\vspace{1pc}
Give the R code for generating $1000$ samples from a normal distribution with mean $1$ and variance $2$, and computing the first sample moment.
\showpoints
\end{document}


