
\documentclass[12pt]{article} % Use A4 paper with a 12pt font size - different paper sizes will require manual recalculation of page margins and border positions

% Generated with LaTeXDraw 2.0.8
% Mon Jun 17 19:00:40 EDT 2013
\usepackage[usenames,dvipsnames]{pstricks}
\usepackage{epsfig}
\usepackage{pst-grad} % For gradients
\usepackage{pst-plot} % For axes
\usepackage[left=1.3cm,right=4.6cm,top=1.8cm,bottom=4.0cm,marginparwidth=3.4cm]{geometry} % Adjust page margins
\usepackage{amsmath} % Required for equation customization
\usepackage{amssymb} % Required to include mathematical symbols
\usepackage{xcolor} % Required to specify colors by name
\usepackage{amsthm}
\usepackage{float}
\usepackage{tikz}
\usetikzlibrary{shapes,backgrounds,trees}
\usepackage{wasysym}

\makeatletter
\newcommand{\mytag}[2]{%
  \text{#1}%
  \@bsphack
  \protected@write\@auxout{}%
         {\string\newlabel{#2}{{#1}{\thepage}}}%
  \@esphack
}
\makeatother

\setlength{\parindent}{0cm} % Remove paragraph indentation
\newcommand{\tab}{\hspace*{2em}} % Defines a new command for some horizontal space
%\newcommand{\choose}[2]{\left(\begin{matrix}
%{#1}\\{#2}
%\end{matrix}\right)}

\title{Introduction to Probability Theory - Lecture 8}
%----------------------------------------------------------------------------------------

\newtheorem{defn}{Definition}
\newtheorem{example}{Example}
\newtheorem{prop}{Proposition}
\newtheorem{exer}{Exercises}
\newtheorem{thm}{Therorem}
\begin{document}
\maketitle
\section{Common Discrete Distributions - Continued}
Last time, we defined random variables with two of the common discrete distributions: namely, $X =$ number of successes in $n$ Bernoulli trials with success probability $p$ - a random variable with the binomial distribution and $X = $ number of items of one type when sampling $n$ items from a total of $w+b$ items, where there are $w$ items of type 1 and $b$ items of type 2 - a random variable with the hypergeometric distribution. we now present a theorem that illustrates a relationship between these two different distributions:\\\\
\begin{thm}
If $X\sim HGeom(w,b,n)$, then for each $x=0,1,2,...,n$ and for $N=w+b. p=\frac{w}N$:
$$\lim\limits_{N\rightarrow\infty}\frac{{w\choose{x}}{{b\choose{n-x}}}}{{{w+b}\choose{n}}} = \lim\limits_{N\rightarrow\infty}\frac{{w\choose{x}}{{{N-w}\choose{n-x}}}}{{{N}\choose{n}}} = {n\choose{x}}p^x\left(1-p\right)^{n-x}$$
\end{thm}
The proof of this theorem is an exercise in manipulation of factorials, and we will not cover it. What is important to realize is the following:\\\\
When $n$ is small relative to $N=w+b$, the binomial distribution is a good approximation for the hypergeometric. That is because when $n$ is small relative to $N$, sampling without replacement looks a lot like sampling with replacement.  This is used a lot in polling, for example. If we poll $1,000$ people out of $10,000,000$, technically speaking, we are sampling \emph{without} replacement. Practically speaking, however, it is about the same as sampling \emph{with} replacement, because the sample is so small, the probability of a person being chosen twice is negligible.\\\\
The book has a nice example relating $HGeom$ to $Bin$ as a result of conditioning. See example $3.9.1$.
\end{document}