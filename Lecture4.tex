
\documentclass[12pt]{article} % Use A4 paper with a 12pt font size - different paper sizes will require manual recalculation of page margins and border positions

% Generated with LaTeXDraw 2.0.8
% Mon Jun 17 19:00:40 EDT 2013
\usepackage[usenames,dvipsnames]{pstricks}
\usepackage{epsfig}
\usepackage{pst-grad} % For gradients
\usepackage{pst-plot} % For axes
\usepackage[left=1.3cm,right=4.6cm,top=1.8cm,bottom=4.0cm,marginparwidth=3.4cm]{geometry} % Adjust page margins
\usepackage{amsmath} % Required for equation customization
\usepackage{amssymb} % Required to include mathematical symbols
\usepackage{xcolor} % Required to specify colors by name
\usepackage{amsthm}
\usepackage{float}
\usepackage{tikz}
\usetikzlibrary{shapes,backgrounds,trees}
\usepackage{wasysym}


\setlength{\parindent}{0cm} % Remove paragraph indentation
\newcommand{\tab}{\hspace*{2em}} % Defines a new command for some horizontal space
%\newcommand{\choose}[2]{\left(\begin{matrix}
%{#1}\\{#2}
%\end{matrix}\right)}

\title{Introduction to Probability Theory - Lecture 3}
%----------------------------------------------------------------------------------------

\newtheorem{defn}{Definition}
\newtheorem{example}{Example}
\newtheorem{prop}{Proposition}
\newtheorem{exer}{Exercises}
\newtheorem{thm}{Therorem}
\begin{document}
\maketitle
\section{Conditional Probability}
Conditional probability is how we incorporate observations. For example, we might want to know the probability that a particular individual has lung cancer. His or her chances vary with certain risk factors. I.e.:
$$P(\textrm{lung cancer})\neq P(\textrm{lung cancer given person is a smoker})$$
\begin{defn}
$$P(A|B) = \frac{P(A\cap B)}{P(B)} \;\;\;\;\;\textrm{provided }BP(B)\neq 0$$
where the left hand side is read 'probability of $A$ given $B$'. 
\end{defn}
Let's focus on the right hand side. We are intersecting the event $A$ with the event $B$, and dividing by $P(B)$. This is essentially restricting the sample space to $B$, and \emph{renormalizing}, so that $P(B|B) = 1$.
\begin{example}
Consider a box containing $100$ microchips. Some are made in factory 1, others in factory 2. Some are defective and others are not. We choose a chip at random. Let $A$ be the event that the chip is defective and $B$ be the event that the chip was made in factory 1 (so that $A^c$ is the event that the chip is not defective and $B^c$ is the event that the chip was made in factory 2). Suppose the following table describes the contents of the box:\\\\
\begin{tabular}{c|c c|c}
& $B$ & $B^c$& \\
\hline
$A$& $15$& $5$ &$20$\\
$A^c$& $45$ & $35$ &$30$\\
\hline
& $60$ & $40$ & $100$
\end{tabular}

What is the probability the chip is defective?
$$P(A) = \frac{20}{100} = 0.20$$

Suppose each chip is labeled with the factory of origin. What is the probability the chip is defective, given that it was made in factory 1?
$$P(A|B) = \frac{15}{60} = 0.25$$
Note that we can calculate this directly by restricting to the column of the table corresponding to $B$, or we can use the definition:
$$P(A|B) = \frac{P(A\cap B)}{P(B)} = \frac{15}{100}/\frac{60}{100} =  \frac{15}{60} = 0.25$$
\end{example}
\end{document}