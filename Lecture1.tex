
\documentclass[12pt]{article} % Use A4 paper with a 12pt font size - different paper sizes will require manual recalculation of page margins and border positions

% Generated with LaTeXDraw 2.0.8
% Mon Jun 17 19:00:40 EDT 2013
\usepackage[usenames,dvipsnames]{pstricks}
\usepackage{epsfig}
\usepackage{pst-grad} % For gradients
\usepackage{pst-plot} % For axes
\usepackage{marginnote} % Required for margin notes
\usepackage{wallpaper} % Required to set each page to have a background
\usepackage{lastpage} % Required to print the total number of pages
\usepackage[left=1.3cm,right=4.6cm,top=1.8cm,bottom=4.0cm,marginparwidth=3.4cm]{geometry} % Adjust page margins
\usepackage{amsmath} % Required for equation customization
\usepackage{amssymb} % Required to include mathematical symbols
\usepackage{xcolor} % Required to specify colors by name
\usepackage{amsthm}
\usepackage{float}


\setlength{\parindent}{0cm} % Remove paragraph indentation
\newcommand{\tab}{\hspace*{2em}} % Defines a new command for some horizontal space


\title{Introduction to Probability Theory - Lecture 1}
%----------------------------------------------------------------------------------------

\newtheorem{defn}{Definition}
\newtheorem{example}{Example}
\newtheorem{prop}{Proposition}
\newtheorem{exer}{Exercises}
\newtheorem{thm}{Therorem}
\begin{document}
\maketitle
\section{Why do we study probability?}
Probability is the mathematical basis of the science of statistics. In particular, there are a couple of probability theorems that form the basis for much statistical inference: The Law of Large Numbers (LLN) and the Central Limit Theorem.

First, let's state an informal version of the LLN:

\begin{thm}
Let $X_1,...,X_n$ be independent samples from a population with mean $\mu$ (unknown) and define:
$$S_n = \frac{X_1+...+X_n}n$$.
Then as $n\rightarrow\infty, S_n\rightarrow\mu$.
\end{thm}
Question: in what sense does $S_n\rightarrow\mu$? What does this mean? What should it mean (intuitively)? 

Now, to really state the LLN, we need to be a bit more precise. 
\begin{thm}
Let $X_1,...,X_n$ be independent and identically distributed random variables with mean $\mu$ and variance $\sigma^2<\infty$. Then as $n\rightarrow\infty$, $S_n\mu:$
$$S_n=\frac{X_1+...+X_n}{n}.$$
\end{thm}
\end{document}