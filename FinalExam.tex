
\documentclass{article}
\usepackage{amsmath}

\oddsidemargin=0in
\evensidemargin=0in
\textwidth=6.3in
\topmargin=-.5in
\textheight=9in

\parindent=0in
\pagestyle{empty}
\newcommand{\pdf}[2]{\left\{\begin{matrix}
{#1} & {#2}\\\\\\0&\textrm{otherwise}
\end{matrix}\right.}
\newcommand{\infi}{\int_{-\infty}^\infty}

\input{testpoints}

\begin{document}


% ooooooooooooooooooooooooooooooooooooooooooooooooooooooooooooooo
%                             COVER
% ooooooooooooooooooooooooooooooooooooooooooooooooooooooooooooooo 

\centerline{\huge \bf Final Exam}        %%%(number of test)
\vfill \vfill
   
Biostat 701                               %%%(class number and section) 

11/18/2015 \hfill                             %%%(date of test)
{\bf Name: } $\underbrace{\hspace{2.7in}}_{\mbox{\tiny by writing my 
                                           name i swear by the honor code}}$
\vfill \vfill \vfill

{\bf Read all of the following information before starting the exam:}
\vspace{1pc}

\begin{itemize}                        %%%(change info. as desired)
	\item  Show all work, clearly and in order, if you want to get full
	credit.  I reserve the right to take off points if I cannot see how you 
	arrived at your answer (even if your final answer is correct).
	
	\item Justify your answers algebraically whenever possible to ensure 
	full credit. 
		
	\item  This test has 11 problems  %%%(number of problems)
	and is worth 100 points.           %%%(insert total number of points)  
	
	\item  Good luck!
\end{itemize}

\vfill \vfill \vfill

\clearpage

% problem with three parts

\begin{problem}{5}

Give a formal definition of a random variable.
\vspace{1pc}

\end{problem}

\begin{problem}{10}

For each of the following distributions, give an example of a random variable that has that distribution. Define the experiment and all relevant parameters.

\begin{enumerate}
\item Bernoulli
\item Binomial
\item Geometric
\item Hypergeometric
\item Negative Binomial
\end{enumerate}

\end{problem}

\begin{problem}{15}
Let $f(x)$ have the following form:
$$f(x) = \pdf{kx^{-(k+1)}}{1<x<\infty}$$

\vspace{1pc}

\newpart{5}
For what values of $k$ is $f$ a valid pdf?

\vspace{1pc}

\newpart{5}
Find the CDF corresponding to the pdf in part a.

\vspace{1pc}

\newpart{5}
For what values of $k$ does $E(X)$ exist?
\end{problem}


\begin{problem}{10}
Let $\psi_X(t)=\log M_X(t)$, where $M_X(t)$ is a MGF. The function $\psi_X(t)$ is called the \textbf{cumulant generating function} of $X$, and the value of the $r$th derivative evaluated at $t=0$, $\kappa_r=\psi^{(r)}_X(0)$, is called the $r$th cumulant of $X$.\\

\newpart{5}

Show that $E(X)=\kappa_1$.

\vspace{1pc}

\newpart{5}

Show that $\text{Var}(X)=\kappa_2$.

\vspace{1pc}

\end{problem}

\begin{problem}{10}
Let $X_1,X_2,X_3,X_4$ be i.i.d with mean $\mu$ and variance $\sigma^2$ Let:
$$Y = X_1 + 2X_2 +X_3 - X_4$$

Find $E(Y)$ and $Var(Y)$.
\end{problem}

\vspace{1pc}

\begin{problem}{10}
Let $X\sim Unif(0,1)$. Use the CDF method to find the pdf of 
$$Y=X^{\frac14}$$  

\end{problem}

\begin{problem}{15}
Let $f(x,y)$ have the following joint pdf:
$$f(x) = \pdf{\frac1\pi e^{-\frac{(x^2+y^2)}2}}{xy>0}$$

\vspace{1pc}

\newpart{5}
Compute the marginals of $X$ and $Y$.

\vspace{1pc}

\newpart{5}
Are $X$ and $Y$ independent? Why or why not?

\vspace{1pc}

\newpart{5}
Is $X,Y\sim BVN$? Why or why not?
\end{problem}

\begin{problem}{5}
The probability that a rare species of hamster will give birth to one male and one female is $1/3$. The probability that the hamster will give birth to a male is $1/2$. What is the probability that the hamster will give birth to a female given that the hamster has given birth to a male? 
\end{problem}


\begin{problem}{10}
Let $X_1,...,X_n$ be iid and let $S_n =X_1+...+X_n$. 

\vspace{1pc}

\newpart{5}
Find $E(X_1|S_n)$. Hint: Use the properties of conditional expectation and the fact that the $X_i$ are identically distributed.

\vspace{1pc}

\newpart{5}

Does the result in a hold if the $X_i$ are \emph{not} independent?
\end{problem}

\begin{problem}{5}
State either the Weak or Strong Law of Large Numbers, and describe what it means informally. 

\vspace{1pc}

\end{problem}

\begin{problem}{5}
State the Central Limit Theorem formally (the version we proved in class), and describe what it means informally. 

\vspace{1pc}

\end{problem}


% ooooooooooooooooooooooooooooooooooooooooooooooooooooooooooooooo
%                         BONUS & SURVEY
% ooooooooooooooooooooooooooooooooooooooooooooooooooooooooooooooo

\showpoints
\end{document}


