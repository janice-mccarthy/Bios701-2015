
\documentclass[12pt]{article} % Use A4 paper with a 12pt font size - different paper sizes will require manual recalculation of page margins and border positions

% Generated with LaTeXDraw 2.0.8
% Mon Jun 17 19:00:40 EDT 2013
\usepackage[usenames,dvipsnames]{pstricks}
\usepackage{epsfig}

\usepackage{pst-grad} % For gradients
\usepackage{pst-plot} % For axes
\usepackage[left=1.3cm,right=4.6cm,top=1.8cm,bottom=4.0cm,marginparwidth=3.4cm]{geometry} % Adjust page margins
\usepackage{amsmath} % Required for equation customization
\usepackage{amssymb} % Required to include mathematical symbols
\usepackage{xcolor} % Required to specify colors by name
\usepackage{amsthm}
\usepackage{float}
\usepackage{tikz}
\usepackage{pgfplots}
\usetikzlibrary{shapes,backgrounds,trees}
\usepackage{wasysym}
\usepackage{scrextend}
\makeatletter
\newcommand{\infi}{\int_{-\infty}^\infty}
\newcommand{\mytag}[2]{%
  \text{#1}%
  \@bsphack
  \protected@write\@auxout{}%
         {\string\newlabel{#2}{{#1}{\thepage}}}%
  \@esphack
}
\makeatother

\setlength{\parindent}{0cm} % Remove paragraph indentation
\newcommand{\tab}{\hspace*{2em}} % Defines a new command for some horizontal space
%\newcommand{\choose}[2]{\left(\begin{matrix}
%{#1}\\{#2}
%\end{matrix}\right)}
\date{}
\title{Introduction to Probability Theory - Lecture 23}
%----------------------------------------------------------------------------------------
\newcommand{\pdf}[2]{\left\{\begin{matrix}
{#1} & {#2}\\\\\\0&\textrm{otherwise}
\end{matrix}\right.}
\newtheorem{defn}{Definition}
\newtheorem{example}{Example}
\newtheorem{prop}{Proposition}
\newtheorem{exer}{Exercises}
\newtheorem{thm}{Therorem}
%\pgfplotsset{compat=1.12}
\begin{document}
\maketitle

\section{Transformations - Continued}
\subsection{Review of Polar Coordinates}
A very common change of coordinates is a change from 'Cartesian' coordinates (just the usual coordinates with horizontal and vertical axes at right angles to one another) to 'polar' coordinates, or coordinates that label a point in the plane using the angle $\theta$ from a fixed axis and the distance $r$ from the origin.\\\\

\begin{tikzpicture}
  \begin{axis}[ 
    xlabel=$x$,
    ylabel={$y$},
    xmin=0,ymin=0,xmax=2,ymax=2,
%    scaled ticks=false,
    xticklabel=\empty,yticklabel=\empty
  ] 
 \draw[dashed] (0,0) to (1cm,2cm) node[left=0.5cm,below=0.25cm]{$r$};
 \draw[fill] (1cm,2cm) circle[radius=0.5mm] node [above] {$(x,y)$};
 \draw[dashed] (1cm,0) to (1cm,2cm) node[below=1.8cm,left=0.38cm] {$\theta$};
 
 \end{axis}
\end{tikzpicture} 

Directly from the diagram, we can see that:

$$x=r\cos(\theta)\;\;\;\;\; y=r\sin(\theta) \;\;\;\;\; r = \sqrt{x^2+y^2}$$

Now, suppose we wish to compute the following definite integral:

$$\int_A f(x,y) dx dy$$

but it is more convenient to change variables to polar coordinates for the computation. We need to compute the Jacobian determinant of the transformation:

$$\begin{matrix}
\frac{\partial x}{\partial r} = \cos(\theta) & \frac{\partial x}{\partial \theta} = -r\sin(\theta)\\\\\frac{\partial y}{\partial r} = \sin(\theta) & \frac{\partial y}{\partial \theta} = r\cos(\theta)
\end{matrix}$$

Then 
$$\mathcal{J} = \left|\begin{matrix}
\cos(\theta) & -r\sin(\theta)\\\\ \sin(\theta) & r\cos(\theta)
\end{matrix}\right| = r\cos^2(\theta) + r\sin^2(\theta) = r$$

Thus, the integral transforms as:

$$\int_A f(x,y) dx dy = \int_{T(A)} f(r,\theta) r dr d\theta$$

Now, there is just one more thing that must be addressed: what is the domain $T(A)$? What we need to do is transform the region $A$ into polar coordinates. This is best explained via example. Suppose we wish to integrate $f$ over a region $A=$ a disk about the origin of radius $5$. To describe such a region in polar coordinates, we must let $0\leq\theta\leq 2\pi$ and $0<r\leq5$. \\\\
\begin{tikzpicture}
  \begin{axis}[ 
    xlabel=$x$,
    ylabel=$y$,
    xmin= -7,ymin= -7,xmax=7,ymax=7
    %   xticklabel=\empty,yticklabel=\empty
  ] 

 \draw (axis cs:0,0) circle(2cm) ;
 \draw[dashed] (axis cs:0cm,0cm) -- (axis cs:5cm,0cm);
 \draw[dashed] (axis cs:0cm,0cm) -- (axis cs:1cm,2cm);
 \node[draw=black,shape=circle] at (axis cs:.5cm,.5cm) {$\theta$};
 \end{axis}

% \node[ {$\theta$}] at axis cs:(1cm,1cm); 
\end{tikzpicture} 


Often, in probability, we integrate over the entire plane $-\infty<x<\infty$ and $-\infty<y<\infty$.  In polar coordinates, this corresponds to letting $0<r<\infty$ and $0\leq \theta\leq 2\pi$. 

\textcolor{red}{Note: these plots need to be edited. I've spent FAR too much time on them, so I am leaving them for now.}\\\\
\subsection{Another Example of 2D Change of Variables}
\begin{example}
Let $X_1$, $X_2$ be exponential random variables with parameter $\lambda = 1$.
$$f_X(x_1,x_2) = \pdf{e^{-(x_1+x_2)}}{x_1>0, x_2>0}$$
Define:
$$Y_1 = X_1$$
$$Y_2 = X_1 + X_2$$
so that:
$$x_1 = y_1 \textrm{ and } x_2 = y_2 - y_1$$
and
$$\mathcal{J} = \left|\begin{matrix}
1&0\\\\-1,1
\end{matrix}\right| = 1$$
The joint pdf of $Y_1,Y_2$ is 
$$f_Y(y_1,y_2) = f_x(y_1,y_2 - y_1) (1) = \pdf{e^{-y_2}}{0<y_1<y_2<\infty}$$
where the limits on $y_1$ and $y_2$ are obtained via
$$x_1> 0 \iff y_1>0$$
and 
$$x_2>0\iff y_2-y_1>0\iff y_2>y_1$$
\end{example}
\subsection{Convolutions}
*** TO BE COMPLETED ***
\end{document}