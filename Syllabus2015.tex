
\documentclass[12pt,a4paper]{article} % Use A4 paper with a 12pt font size - different paper sizes will require manual recalculation of page margins and border positions

% Generated with LaTeXDraw 2.0.8
% Mon Jun 17 19:00:40 EDT 2013
\usepackage[usenames,dvipsnames]{pstricks}
\usepackage{epsfig}
\usepackage{pst-grad} % For gradients
\usepackage{pst-plot} % For axes
%\usepackage{marginnote} % Required for margin notes
%\usepackage{wallpaper} % Required to set each page to have a background
%\usepackage{lastpage} % Required to print the total number of pages
\usepackage[left=1.3cm,right=4.6cm,top=1.8cm,bottom=4.0cm,marginparwidth=3.4cm]{geometry} % Adjust page margins
\usepackage{amsmath} % Required for equation customization
\usepackage{amssymb} % Required to include mathematical symbols
\usepackage{xcolor} % Required to specify colors by name
\usepackage{amsthm}
\usepackage{float}


\setlength{\parindent}{0cm} % Remove paragraph indentation
\newcommand{\tab}{\hspace*{2em}} % Defines a new command for some horizontal space


\title{BIOSTAT 701: Introduction to Probability Theory }
%--------------------------------------------------------------------------------
\begin{document}
\maketitle

\noindent \textbf{Instructor}: Janice McCarthy\\
\noindent \textbf{Office}: Hock Plaza, Room 11102\\
\noindent \textbf{Email Address} Janice.McCarthy@duke.edu\\
\noindent \textbf{Office Phone:} 919-688-5891\\
\noindent \textbf{Textbook:} Introduction to Probability, J. Blitzstein and J. Hwang, Publisher: CRC Press\\
\noindent \textbf{Days and Times:} M/W 10:05-11:20\\
\noindent \textbf{Place:} Hock 11025\\
\noindent \textbf{Office Hours:} M/W 12-2pm\\

\noindent Please feel free to email or stop by my office at any time. If I am in my office, I am generally available for questions.\\\\
\noindent\textbf{Course Objectives:}\\
This is a course in the fundamentals of probability theory. \emph{This is a theory course.} As such, students will be required to perform some simple derivations, understand and apply formal definitions and use notions from former mathematics courses, including multivariate calculus. However, examples will often attempt to connect theory to the practice of biostatistics. By the end of the course, students should be able to:
\begin{itemize}
\item Explain the concepts and formal definitions of probability and conditional probability.
\item Understand the concepts of discrete and continuous random variables, their probability distributions, expectation, variance, etc.
\item Solve problems related to joint probability distributions, independence and conditional probability distributions.
\item Derive expectations, correlations, conditional expectations and moment generating functions of random variables.
\item Derive the distribution of functions of random variables. In particular, derive the distributions of common functions of normal random variables such as $\chi^2$.
\end{itemize}

\noindent \textbf{Course Organization:}\\
\noindent We will follow the general organization of the textbook with additional notes and examples. The student is responsible for all work covered in lectures.\\
\\
\noindent\textbf{Homework:}\\
\noindent Homework will be assigned roughly after each chapter of the text is completed. Late homework will not be accepted or graded unless the student can show sufficient cause due to health, family or other unavoidable circumstances. Students are encouraged to work in groups, but your solutions must be your own and not copied from another student.\\\\
\noindent\textbf{Class Organization:}\\
\noindent We will begin each class with a lecture, followed by a 15-20 minute problem session, where students will work in small groups on 'by hand' problems or with the computer for R samples and exercises. \textbf{A laptop is required equipment for every class meeting.}\\\\
\noindent\textbf{Grading:}\\
\noindent\textbf{The following is an APPROXIMATE grade breakdown. The instructor reserves the right to modify the algorithm.} Note that this is often to the benefit of the student.\\\\

\hspace{1.5in}
\textbf{Mid-term Exam 1} 30 \%

\hspace{1.5in}  
\textbf{Mid-term Exam 2} 30 \%

\hspace{1.5in}  
\textbf{Quizzes} 3\%

\hspace{1.5in}
\textbf{Homework} 2\%

\hspace{1.5in}
\textbf{Final Exam} 35 \%

\end{document}